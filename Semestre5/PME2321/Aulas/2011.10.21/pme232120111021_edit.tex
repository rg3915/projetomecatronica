\documentclass[a4paper,12pt]{memoir}
\usepackage[utf8]{inputenc}
\usepackage{../../../../preambulo}

% estilo de capitulo
\corpedersen
\chapterstyle{pedersen}

\title{PME2321}
\author{Vitor M. Martins\\ R\'egis S. Santos}
\date{}
\newcommand{\subtitle}{Notas de Aula do Curso PME2321}
%*****************************************************
\begin{document}

\begin{titlingpage}
  \maketitle
\end{titlingpage}

\tableofcontents

\chapter{Lista de Exercícios}

\section{Exercícios}

\begin{Exerc}[9.69 6\pa Ed.]
Considere a figura a seguir:

% \myfig[scale=0.58]{1}{1}

\tkzfig{fig01}{fig01}
 
\begin{enumerate}[a)]
  \item $T_2 =$
  \item $m_2 =$
  \item $P_3 =$
  \item $T_{2 \imp 3} =$
\end{enumerate}
\end{Exerc}

\begin{sol}
\begin{enumerate}[a)]
 \item \textbf{Processo Transiente} (regime uniforme) Volume de Controle $\imp$ mina $+$ compressor

1ª Lei: 

\[
m_e h_e = m_2 u_2 - m_1 u_1 + W_c \footnote{Trabalho do compressor}
\]

\[
m_2 = m_1 + m_e
\]

2ª Lei:

\[
\begin{gathered}
  m_2 s_2 - m_1 s_1 = m_e s_e \hfill \\
  s_e = s_1 \hfill \\ 
  m_2 = m_1 + m_e \hfill \\ 
  s_e = s_1 \hfill \\ 
\end{gathered} 
\]

\[
\begin{gathered}
  s_2 - s_{1} = 0 = (s_{T2}^0 - s_{T1}^0 - R\ln(\frac{P_2}{P_1})) \hfill \\
  0 = (s_{T2}^0 - 6.83512 - 0.287\ln(\frac{2100}{100})) \hfill \\ 
  s_{T2}^0 = 7.709 \imp T_2 = 680\,K \hfill \\ 
\end{gathered} 
\]

\item

\[
\begin{gathered}
  P_2 V = m_2 RT_2 \hfill \\
  m_2 = 1.0760*10^6\,Kg \hfill \\ 
\end{gathered} 
\]

\item

\[
\begin{gathered}
  \frac{P_2}{T_2} = \frac{P_3}{T_3} \hfill \\
  P_3 = 1235\,KPa \hfill \\ 
\end{gathered} 
\]

\item Sistema:

\[
Q_{2 \imp 3} = U_3 - U_2 + W_{2 \imp 3} \footnote{Trabalho nulo}
\]

\[
Q_{2 \imp 3} = m_2 (u_3 \footnote{$u_3 [T_3 = 400\,K]$} - u_2 \footnote{$u_2 [T_2 = 680\,K]$}) = -2,264.10^8\,KJ
\]

\end{enumerate}
\end{sol}

\begin{Exerc}[8.135 6\pa Ed.]
Considere a figura a seguir:

% \myfig[scale=0.58]{2}{2}
\tkzfig{fig02}{fig02}

\begin{enumerate}[a)]
 \item $W =$
 \item Isso é possível?
\end{enumerate}
\end{Exerc}

\begin{sol}

\begin{enumerate}[a)]
 \item 1ª Lei

\[
\begin{gathered}
  Q_{1 \imp 2S} = m_2 u_{2S} - m_1 u_1 + W_{1 \imp 2S} \hfill \\
  W_{1 \imp 2S} = mc_V (T_1 - T_{2,S}) \hfill \\ 
\end{gathered} 
\]

2ª Lei (Adiabático e Reversível)

\[
s_2 - s_1 = \frac{Q_{1 \imp 2S}}{T}
\]

Portanto: $s_2 = s_1$

Hipótese: $c_p$ constante

\[
\frac{T_{2,s}}{T_1} = (\frac{P_2}{P_1})^{\frac{k-1}{k}}
\]

Portanto: $T_{2,S} = 335.3\,K$

 \item Possível?

\[
\begin{gathered}
  \Delta S_{\operatorname{liq}} > 0 \hfill \\
  \Delta S_{\operatorname{liq}} = (m_2 s_2 - m_1 s_1)\footnote{sistema} - \frac{Q_{1 \imp 2S}}{T_0}\footnote{meio} \hfill \\ 
  \Delta S_{\operatorname{liq}} = (0.002094) - \frac{-0.5774}{293} = 0.004065\,kJ/K \hfill \\
  \Delta S_{\operatorname{liq}} = m(s_2 - s_1) = m[c_p \ln(\frac{T_2}{T_1}) - R \ln(\frac{P_2}{P_1})] \hfill \\
  \Delta S_{\operatorname{liq}} = 0.002094\,kJ/k \hfill \\
\end{gathered} 
\]

\end{enumerate}
\end{sol}

\begin{Exerc}[8.117 6\pa Ed.]
Considere a figura a seguir:

% \myfig[scale=0.58]{3}{3}
\tkzfig{fig03}{fig03}

\begin{enumerate}[a)]
 \item $W =$
 \item $q =$
 \item $s_{\operatorname{ger}} =$
\end{enumerate}
\end{Exerc}

\begin{sol}
Hipóteses

\begin{itemize}
\item $c_{P}$ constante
\item $Pv = RT$
\item $PV^{n}=cte, n = 1.3$
\end{itemize}

\begin{enumerate}[a)]
 \item trabalho específico

\[
\begin{gathered}
  w_{1 \imp 2} = \frac{P_2 v_2 - P_1 v_1}{1-n} \hfill \\
  w_{1 \imp 2} = \frac{R(T_2 - T_1)}{1-n} \hfill \\
  w_{1 \imp 2} = -191.3\,kJ/kg \hfill \\
\end{gathered} 
\]

 \item

\[
\begin{gathered}
  q_{1 \imp 2} = u_2 - u_1 + w_{1 \imp 2} \hfill \\
  q_{1 \imp 2} = c_V (T_2 - T_1) + w_{1 \imp 2} \hfill \\ 
  q_{1 \imp 2} = -48.03\,kJ/kg \hfill \\ 
\end{gathered} 
\]

 \item

\[
\begin{gathered}
  s_{\operatorname{ger}} = (s_2 - s_1) - \frac{q _{1 \imp 2}}{T_0} = 0.0037\,kJ/kg \hfill \\
  s_2 - s_1 = c_p \ln(\frac{T_2}{T_1}) - R\ln(\frac{P_2}{P_1}) \hfill \\ 
\end{gathered} 
\]

\end{enumerate}
\end{sol}

\begin{Exerc}[8.117 6\pa Ed.]
Considere a figura a seguir:

% \myfig[scale=0.58]{4}{4}
\tkzfig{fig04}{fig04}

\end{Exerc}

\begin{sol}

% \myfig[scale=0.38]{5}{5}
\tkzfig{fig05}{fig05}

Sistema: água

1ª Lei: 

\[
\begin{gathered}
  Q_{1 \imp 2} = m(u_2 - u_1) + W_e \hfill \\
  W_e = P_e (v_2 - v_1) m = -966.7\,kJ \hfill \\ 
\end{gathered} 
\]

\begin{itemize}
\item $m = 2.294\,kg$
\item $v_2 = 0.001452\,m^3/kg$
\item $v_1 = 0.04359\,m^3/kg$
\item $u_1 = 3433\,kJ/kg$
\item $u_2 = 1393\,kJ/kg$

\[
Q_{1 \imp 2} = -5647\,kJ
\]

Processo Global Reversível

\[
\begin{gathered}
  \Delta S_{\text{eq}} = 0 \hfill \\
  \Delta S_{\text{eq}} = \Delta S_{\text{SIST}} + \Delta S_{\text{MEIO}} \hfill \\
  \Delta S_{\text{SIST}} = m(s_2 - s_1) \hfill \\
  \Delta S_{\text{MEIO}} = -\frac{Q_2}{T_0} \hfill \\ 
\end{gathered} 
\]

\end{itemize}

\end{sol}

\end{document}