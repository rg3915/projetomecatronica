\documentclass[a4paper]{article}
\usepackage[utf8]{inputenc}
\usepackage[T1]{fontenc}
\usepackage[brazil]{babel}
\usepackage[margin=3cm]{geometry}
\usepackage{xcolor,showexpl}	%exibe os codigos
\usepackage{hyperref}

\usepackage{listings}
\lstset{explpreset={
    language=bash,
    float=hbp,
    basicstyle=\ttfamily\small, 
    identifierstyle=\color{white}, 
    keywordstyle=\color{white}, 
    stringstyle=\color{white}, 
    commentstyle=\color{white}, 
    columns=flexible, 
    tabsize=4, 
    frame=none,   %%frame=single
    rframe={}
    },
    language=bash,
    basicstyle=\ttfamily\small, 
    identifierstyle=\color{white}, 
    keywordstyle=\color{white}, 
    stringstyle=\color{white}, 
    commentstyle=\color{white}, 
    hsep=10mm,
    extendedchars=true, 
    showspaces=false, 
    showstringspaces=false, 
    numbers=none,  %%numbers=left
    %numberstyle=\tiny, 
    breaklines=true, 
    backgroundcolor=\color{gray}, 
    breakautoindent=true, 
    captionpos=b,
    xleftmargin=0pt,
}

\title{git tutorial}
\author{R\'egis da Silva Santos}
\date{21 de dezembro de 2011}

\begin{document}

\maketitle

\section*{Introdução ao git}

% \texttt{joaopcop@gmail.com}

% todos os paragrafos sem identação
\setlength\parindent{0pt}

Veja no site a seguir como \emph{instalar} e \emph{configurar} o \textbf{git}:

\url{http://codexico.com.br/blog/linux/tutorial-simples-como-usar-o-git-e-o-github/}

Leia também \url{http://progit.org}.

Para começar o git abra o \emph{terminal} na pasta do seu repositório, por exemplo, \emph{meugit}.

iniciando o git
\begin{lstlisting}
git init
\end{lstlisting}

mostrando o estado do repositório
\begin{lstlisting}
git status
\end{lstlisting}

comentário inicial do repositório
\begin{lstlisting}
git commit
\end{lstlisting}

comentários
\begin{lstlisting}
git commit --allow-empty commit -m ``vazio''
git commit -a -m ``comentario''
\end{lstlisting}

mostrando o histórico de commits
\begin{lstlisting}
git log
\end{lstlisting}

configuração
\begin{lstlisting}
git config --global user.name ``seu nome''
git config --global user.email ``seu email''
\end{lstlisting}

\subsection*{Criando e adicionado o primeiro arquivo}

Agora vamos criar um arquivo qualquer e verificar seu status
\begin{lstlisting}
touch arquivo01
git status
\end{lstlisting}

Usando a opção \verb|-s| podemos ver que o arquivo foi adicionado e modificado
\begin{lstlisting}
git status -s
\end{lstlisting}

\textbf{Adicionando} o arquivo efetivamente no repositório
\begin{lstlisting}
git add arquivo01
\end{lstlisting}

Adicionando \textbf{todos} os arquivos
\begin{lstlisting}
git add -A
\end{lstlisting}

Comentando
\begin{lstlisting}
git commit -m ``mensagem''
\end{lstlisting}

\subsection*{Chave}

Chave pública
\begin{lstlisting}
ssh-keygen -t rsa
\end{lstlisting}

A chave encontra-se na pasta \emph{HOME/.ssh/id\_rsa.pub}.

\subsection*{Clonando repositório}

Repositórios a serem clonados
\begin{lstlisting}
git clone git@github.com:vitormmartins/projetomecatronica.git
git clone git@github.com:rg3915/LaTeX.git
\end{lstlisting}

\textbf{Baixando} as atualizações do projeto
\begin{lstlisting}
git pull
\end{lstlisting}

\textbf{Enviando} as alterações
\begin{lstlisting}
git push
\end{lstlisting}

ou
\begin{lstlisting}
git push  origin master
\end{lstlisting}

Adicionando repositório
\begin{lstlisting}
git remote add origin git@github.com:vitormmartins/projetomecatronica.git
\end{lstlisting}

\subsection*{Tags}

Criando tag
\begin{lstlisting}
git tag -a v1.0 -m 'versao 1.0'
git push --tags
\end{lstlisting}

Deletando tag
\begin{lstlisting}
git tag -d v1.0
git push origin :refs/tags/v1.0
\end{lstlisting}

\subsection*{Branchs}

Criando branch
\begin{lstlisting}
git branch mybranch
\end{lstlisting}

Mudando de branch
\begin{lstlisting}
git checkout mybranch
git status
\end{lstlisting}

Mostrando os arquivos do \emph{master}
\begin{lstlisting}
git checkout master
\end{lstlisting}

Mostrando novamente os arquivos do \emph{mybranch}
\begin{lstlisting}
git checkout mybranch
\end{lstlisting}

Voltando para o \emph{git master}
\begin{lstlisting}
git checkout master
\end{lstlisting}

Mostrando todas as branch
\begin{lstlisting}
gitk --all
\end{lstlisting}

\newpage 

Como juntar as branch na \emph{master}?
\begin{lstlisting}
git merge master
\end{lstlisting}

Colocando branch no mesmo nível
\begin{lstlisting}
git rebase mybranch
\end{lstlisting}

Resolvendo conflito de merge
\begin{lstlisting}
git mergetool
\end{lstlisting}

\subsection*{visualizador gráfico do repositório}

\begin{lstlisting}
gitk
\end{lstlisting}

Caso não esteja instalado você pode instala-lo pelo apt
\begin{lstlisting}
sudo apt-get install gitk
\end{lstlisting}

\end{document}
